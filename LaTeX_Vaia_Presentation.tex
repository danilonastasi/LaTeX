\documentclass{beamer}
\usepackage{color}
\usetheme{Berlin} % Frankfurt Madrid Berkeley
\usecolortheme{spruce} % dove crane spruce

\title{Stima del danno forestale a seguito della tempesta Vaia in Italia nell'area compresa tra Bolzano (BZ) e Predazzo (TN)}
\author{Danilo Nastasi}

\begin{document}

\maketitle

\AtBeginSection[] % Do nothing for \section*
{	
\begin{frame}
\frametitle{Outline}
\tableofcontents[currentsection,currentsubsection,currentsubsubsection]
\end{frame}
}


\section{Introduzione}
\begin{frame}
\frametitle{Vaia vista dal satellite - 29/10/2018}
\centering
\includegraphics[width=0.7\textwidth]{Vaia.png} \\
\tiny{\textit{immagine da Chirici et al., 2019 - \url{https://doi.org/10.3832/efor3070-016}}}
\end{frame}

\begin{frame}
\frametitle{Provincia di Bolzano}
\centering
\includegraphics[width=0.9\textwidth]{Bolzano.png} \\
\tiny{\textit{foto da Chirici et al., 2019 - \url{https://doi.org/10.3832/efor3070-016}}}
\end{frame}

\begin{frame}
\frametitle{In basso l'Area di Studio}
\centering
\includegraphics[width=0.9\textwidth]{00mapVaiaOSMDeAgostini.png}
\end{frame}

\begin{frame}
\frametitle{Blue, Green, Red, NIR spectral bands}
\centering
\includegraphics[width=0.8\textwidth]{sentinel_bands.png}
\end{frame}

\begin{frame}
\frametitle{Vegetation reflectance}
\centering
\includegraphics[width=0.6\textwidth]{riflettanza.png} \\
\includegraphics[width=0.35\textwidth]{piante_salute.png} \\
\tiny{\textit{immagini da \url{https://www.agricolus.com/indici-vegetazione-ndvi-ndmi-istruzioni-luso/}}}
\end{frame}

\begin{frame}
\frametitle{Interpretazione dei Valori NDVI}
\centering
\includegraphics[width=0.55\textwidth]{NDVI_range.png} \\
\tiny{\textbf{valori compresi tra -1 e 0 sono tipici di aree non coltivate come corsi d’acqua e zone antropiche.} \\
\textit{Tabella da \url{https://www.agricolus.com/indici-vegetazione-ndvi-ndmi-istruzioni-luso/}}}
\end{frame}

\begin{frame}
%\frametitle{Band Combinations - Natural Color (nc), Color Infrared (cir)}
\centering
\includegraphics[width=0.6\textwidth]{Common Landsat Band Combos_50.png}
\end{frame}

\section{Materiali e Metodi}
\begin{frame}
\frametitle{Strumenti utilizzati}
\centering
\includegraphics[width=0.4\textwidth]{QGIS.png}
\includegraphics[width=0.4\textwidth]{Copernicus_Sentinel.png} \\
\includegraphics[width=0.3\textwidth]{Asus.png}
\includegraphics[width=0.3\textwidth]{R_Language.png}
\end{frame}

\begin{frame}
\frametitle{ESA - Selezione area e periodo di interesse}
\centering
\includegraphics[width=0.8\textwidth]{copernicus_area_search01_02.png}
\end{frame}

\begin{frame}
\frametitle{ESA - Download immagini 26/09/2018}
\centering
\includegraphics[width=0.8\textwidth]{copernicus20180926select.png}
\end{frame}

\begin{frame}
\frametitle{ESA - Download immagini 16/09/2019}
\centering
\includegraphics[width=0.8\textwidth]{copernicus20190916select.png}
\end{frame}

\begin{frame}
\frametitle{QGIS - Immagine Sentinel = 12056 km\textsuperscript{2}}
\centering
\includegraphics[width=0.8\textwidth]{qgisareafull.png}
\end{frame}

\begin{frame}
\frametitle{QGIS - Selezione Area di Studio = 360.77 km\textsuperscript{2}}
\centering
\includegraphics[width=0.8\textwidth]{qgisareaselect.png}
\end{frame}

\begin{frame}
\frametitle{QGIS - Esporta nuova immagine}
\centering
\includegraphics[width=0.8\textwidth]{qgisareaselectexport_02.png}
\end{frame}

\begin{frame}
\frametitle{QGIS - Salvataggio immagine tif}
\centering
\includegraphics[width=0.7\textwidth]{qgisareaselectexport_options_temp.png} \\
operazione da ripetere con tutte le bande salvando i dati in un'\textbf{unica cartella di lavoro}
\end{frame}

\begin{frame}
\frametitle{QGIS - Nuova immagine georeferenziata}
\centering
\includegraphics[width=0.8\textwidth]{qgisareacrop.png}
\end{frame}

\begin{frame}
\frametitle{Formule utilizzate}
% \centering
% \includegraphics[width=0.3\textwidth]{img.jpg} \\
\begin{equation}
 DVI = {REF\_nir -  REF\_red} 
 \end{equation}
 
 \begin{equation}
 NDVI = \frac{REF\_nir - REF\_red}{REF\_nir + REF\_red}
\end{equation}
\begin{itemize}
\item \textbf{DVI} (Difference Vegetation Index)
\item \textbf{NDVI} (Normalized Difference Vegetation Index)
\item \textbf{REF} (Reflectance)
\end{itemize}
\end{frame}

% \begin{equation}
%  DVI\_diff = {DVI2018 -  DVI2019}
% \end{equation}
% \end{frame}

% \begin{equation}
%  NDVI\_diff = {NDVI2018 -  NDVI2019}
% \end{equation}
% \end{frame}

\section{R-Code e Risultati}
\begin{frame}
\frametitle{Librerie utilizzate}
\centering
\includegraphics[width=0.7\textwidth]{R_packages.png}
\end{frame}

\begin{frame}
\frametitle{Definiamo la fuzione}
\centering
\includegraphics[width=0.7\textwidth]{R_Function.png}
\end{frame}

\begin{frame}
\frametitle{Import data - Multi-layer Raster object - 8 bit}
\centering
\includegraphics[width=0.8\textwidth]{R_import.png}
\end{frame}

\begin{frame}
\frametitle{Natural Color(nc) and Color Infrared(cir)}
\centering
\includegraphics[width=0.8\textwidth]{R_ggplot_nc_cir.png}
\end{frame}

\begin{frame}
\frametitle{Natural Color(nc) and Color Infrared(cir) plot}
\centering
\includegraphics[width=0.7\textwidth]{01ggpnc2018_2019.png} \\  \includegraphics[width=0.7\textwidth]{02ggpcir2018_2019.png}
\end{frame}

\begin{frame}
\frametitle{DVI and NDVI}
\centering
\includegraphics[width=0.6\textwidth]{R_DVI.png} \\
\includegraphics[width=0.6\textwidth]{R_NDVI.png}
\end{frame}

\begin{frame}
\frametitle{DVI and NDVI plot}
\centering
\includegraphics[width=0.4\textwidth]{03ggpdvi_2018.png}
\includegraphics[width=0.4\textwidth]{03ggpdvi_2019.png} \\
\includegraphics[width=0.4\textwidth]{04ggpndvi_2018.png}
\includegraphics[width=0.4\textwidth]{04ggpndvi_2019.png}
\end{frame}

\begin{frame}
\frametitle{DVI and NDVI 2018-19 difference}
\centering
\includegraphics[width=0.7\textwidth]{R_DVI_dif.png} \\
\includegraphics[width=0.7\textwidth]{R_NDVI_dif.png}
\end{frame}

\begin{frame}
\frametitle{DVI and NDVI 2018-19 difference plot}
\centering
\includegraphics[width=0.45\textwidth]{03ggpdvi2018_19_diff.png}
\includegraphics[width=0.45\textwidth]{04ggpndvi2018_19_diff.png}
\end{frame}

\begin{frame}
\frametitle{Classification NDVI 2018-19 difference}
\centering
\includegraphics[width=0.9\textwidth]{R_class.png}
\end{frame}

\begin{frame}
\frametitle{Classification NDVI 2018-19 difference plot}
\centering
\includegraphics[width=0.6\textwidth]{05ggpndvidiff_class.png}
\includegraphics[width=0.3\textwidth]{class_freq_02.png}
\end{frame}

\begin{frame}
\frametitle{Elaborazione dati - Tabella}
\centering
\includegraphics[width=0.7\textwidth]{R_table.png}
\end{frame}

\begin{frame}
\frametitle{Tabella dei risultati}
\centering
\includegraphics[width=0.6\textwidth]{table_results.png}
\end{frame}

\begin{frame}
\frametitle{Riepilogo}
\centering
\includegraphics[width=0.45\textwidth]{01ggpnc2018_2019.png}
\includegraphics[width=0.25\textwidth]{04ggpndvi_2018.png}
\includegraphics[width=0.25\textwidth]{04ggpndvi_2019.png} \\
\includegraphics[width=0.45\textwidth]{02ggpcir2018_2019.png}
\includegraphics[width=0.25\textwidth]{04ggpndvi2018_19_diff.png}
\includegraphics[width=0.25\textwidth]{05ggpndvidiff_class.png}
\end{frame}

\section{References}
\begin{frame}
\frametitle{References}
\begin{itemize}
\scriptsize{}
\item \scriptsize{\url{https://www.rdocumentation.org/}}
\item \scriptsize{Chirici, G., Giannetti, F., Travaglini, D., Nocentini, S., Francini, S., D'Amico, G., Calvo, E., Fasolini, D., Broll, M., Maistrelli, F., Tonner, J., Pietrogiovanna, M., Oberlechner, K., Andriolo, A., Comino, R., Faidiga, A., Pasutto, I., Carraro, G., Zen, S., Marchetti, M. (2019), Forest damage inventory after the "Vaia" storm in Italy. Forest - Rivista di Selvicoltura ed Ecologia Forestale, vol.16, pp. 3-9. \url{https://doi.org/10.3832/efor3070-016}}
\item \scriptsize{\url{https://www.agricolus.com/indici-vegetazione-ndvi-ndmi-istruzioni-luso/}}
\end{itemize}
\end{frame}

\end{document}
